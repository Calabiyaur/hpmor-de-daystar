% Schneefl0cke
% \emph{Anmerkung des Übersetzers:}

% Dies ist der Clou, und der Höhepunkt der Geschichte. Alles um euch auf diese Prüfung vorzubereiten, zum Nachdenken zu bringen und Sie zu meistern wurde geschrieben. Hinweise und Hilfsmittel sind in der ganzen Geschichte versteckt…. und nun: \emph{die Abschlussprüfung} nach den Regeln des Original Autors \emph{E.Y.}

% \emph{\hfill\break \emph{Dies ist deine letzte Prüfung.}}

% Du hast 60 Stunden Zeit. Deine Lösung muss es Harry zumindest ermöglichen, dem sofortigen Tod zu entgehen, obwohl er nackt ist, nur seinen Zauberstab in der Hand hält und 36 Todessern plus dem vollständig wiederauferstandenen Lord Voldemort gegenübersteht.

% \emph{Wird keine brauchbare Lösung eingereicht, endet die Geschichte traurig und kürzer.}

% \emph{\hfill\break Beachtet Folgende Einschränkungen:

% 1. Harry muss es aus eigener Kraft schaffen. Die Kavallerie wird nicht kommen. Jeder, der Harry helfen möchte, denkt, er sei bei einem Quidditch-Spiel.}

% 2. Harry darf nur Fähigkeiten einsetzen, die ihm die Geschichte bereits gezeigt hat; er kann nicht in den nächsten 60 Sekunden wortlose Legilimenz entwickeln.

% 3. Voldemort ist böse und kann nicht dazu überredet werden, gut zu sein; die Nutzenfunktion des Charkaters des Dunklen Lords als Antagonist kann nicht verändert werden, indem man mit ihm spricht.

% 4. Wenn Harry seinen Zauberstab erhebt oder in etwas anderem als Parsel spricht, werden die Todesser sofort auf ihn schießen.

% 5. Wenn die einfachste Zeitlinie sonst eine ist, in der Harry stirbt - wenn Harry seinen Zeitumkehrer nicht ohne die Hilfe des Zeitumkehrer erreichen kann - dann kommt der Zeitumkehrer nicht ins Spiel.

% 6. Es ist unmöglich in Parsel Lügen zu erzählen.

% Innerhalb dieser Einschränkungen ist es Harry erlaubt, sein volles Potenzial als Rationalist zu erreichen, jetzt in diesem Moment oder nie, unabhängig von seinen früheren Fehlern.

% Natürlich ist „die rationale Lösung“, wenn man das Wort „rational“ richtig verwendet, nur eine unnötig ausgefallene Art zu sagen „die beste Lösung“ oder „die Lösung, die ich mag“ oder „die Lösung, die ich denke, dass wir verwenden sollten“, und man sollte normalerweise stattdessen einen der letzteren Begriffe sagen.

% (Wir brauchen das Wort 'rational' nur, um über Denkweisen zu sprechen, die unabhängig von bestimmten Lösungen betrachtet werden.)

% Und nach dem Vinge-Prinzip gilt: Wenn man genau weiß, was ein kluger Kopf tun würde, muss man selbst mindestens so klug sein. Jemanden zu fragen „Was würde ein optimaler Spieler für den besten Zug halten?“ sollte keine bessere Antwort hervorbringen als „Was denkst du ist der beste Zug?“ Was ich also in der Praxis meine, wenn ich sage, dass Harry sein volles Potenzial als Rationalist ausschöpfen darf, ist, dass Harry dieses Problem so lösen darf, wie DU es lösen würdest.

% Wenn du mir genau sagen kannst, wie ich etwas tun soll, dann ist es Harry erlaubt, daran zu denken. Aber es dient nicht als Lösung, wenn man z. B. sagt:

% „Harry sollte Voldemort überreden, ihn aus der Falle zu lassen“, wenn man selbst nicht herausfinden kann, wie.

% Ich erinnere euch nochmals daran, dass ihr Stunden zum Nachdenken habt. Nutzt die "Keine Lösungen vorschlagen„ Technik.

% \emph{Und ganz ehrlich, ich meine es ernst, Harry kann in den nächsten 60 Sekunden keine neuen magischen Kräfte entwickeln oder die zuvor genannten Beschränkungen überwinden.}

% \emph{Noch eine Anmerkung des Übersetzers:}

% Es wurde innerhalb der vorgegebenen Zeit tatsächlich eine Lösung gefunden. Und da ich die Geschichte nicht selbst geschrieben habe oder umschreiben werde, wird es auf jeden Fall mit dem „Guten Ende“ weitergehen, auch deshalb weil E.Y. Das schlechte nie veröffentlicht hat.

% Trotzdem werde ich das nächste Kapitel erst in ein paar Tagen veröffentlichen, damit ihr ein bisschen Zeit zum Rätseln habt. Solltet ihr eine Lösung haben, schreibt Sie doch gerne als Kommentar, ich bin sehr gespannt auf eure Ideen!







\chapter{Final Exam}

\section{The following was posted at the end of Chapter 113:}

{\setlength{\parindent}{0pt}
\setlength{\parskip}{.5\baselineskip}

This is your final exam.

You have 60 hours.

Your solution must at least allow Harry to evade immediate death,
despite being naked, holding only his wand, facing 36 Death Eaters
plus the fully resurrected Lord Voldemort.

\textbf{If a viable solution is posted before \emph{12:01\am Pacific Time} (8:01\am UTC) on Tuesday, March 3rd, 2015, the story will continue to Ch.~121.}

\textbf{Otherwise you will get a shorter and sadder ending.}

Keep in mind the following:
\begin{enumerate}
\item Harry must succeed via his own efforts. The cavalry is not coming.
Everyone who might want to help Harry thinks he is at a Quidditch game.
\item Harry may only use capabilities the story has already shown him to
have; he cannot develop wordless wandless Legilimency in the next 60 seconds.
\item Voldemort is evil and cannot be persuaded to be good; the Dark Lord’s utility function cannot be changed by talking to him.
\item If Harry raises his wand or speaks in anything except Parseltongue, the Death Eaters will fire on him immediately.
\item If the simplest timeline is otherwise one where Harry dies—if Harry cannot reach his Time-Turner without Time-Turned help—then the Time-Turner will not come into play.
\item It is impossible to tell lies in Parseltongue.
\end{enumerate}
{\setlength{\parindent}{0pt}
\setlength{\parskip}{.5\baselineskip}
Within these constraints, Harry is allowed to attain his full potential as a rationalist, now in this moment or never, regardless of his previous flaws.

Of course ‘the rational solution’, if you are using the word ‘rational’ correctly, is just a needlessly fancy way of saying ‘the best solution’ or ‘the solution I like’ or ‘the solution I think we should use’, and you should usually say one of the latter instead. (We only need the word ‘rational’ to talk about ways of thinking, considered apart from any particular solutions.)

And by Vinge’s Principle, if you know exactly what a smart mind would do, you must be at least that smart yourself. Asking someone “What would an optimal player think is the best move?” should produce answers no better than “What do you think is best?”

So what I mean in practice, when I say Harry is allowed to attain his full potential as a
rationalist, is that Harry is allowed to solve this problem the way \emph{you} would solve it. If you can tell me exactly how to do something, Harry is allowed to think of it.

But it does not serve as a solution to say, for example, “Harry should persuade Voldemort to let him out of the box” if you can’t yourself figure out how.

\textbf{The rules on Fanfiction dot Net allow at most one review per chapter. Please submit \emph{only one} review of Ch.~113, to submit one suggested solution.}

For the best experience, if you have not already been following Internet conversations about recent chapters, I suggest \textbf{not} doing so, trying to complete this exam on your own, not looking at other reviews, and waiting for Ch.~114 to see how you did.

I wish you all the best of luck, or rather the best of skill.

Ch.~114 will post at \textbf{10\am Pacific (6\pm UTC) on Tuesday, March 3rd, 2015.}

ADDED:

If you have pending exams, then even though the bystander effect is a thing,
I expect that the collective effect of ‘everyone with more urgent life issues stays out of the effort’ shifts the probabilities very little (because diminishing marginal returns on more eyes and an already-huge population that is participating).

\textbf{So if you can’t take the time, then please don’t.} Like any author, I enjoy the delicious taste of my readers’ suffering, finer than any chocolate; but I don’t want to \emph{hurt} you.

Likewise, if you hate hate hate this sort of thing, then don’t participate! Other people ARE enjoying it. Just come back in a few days. I shouldn’t even need to point this out.

I remind you again that you have hours to think. Use the Hold Off On Proposing Solutions, Luke.

And really truly, I do mean it, Harry cannot develop any new magical powers or transcend previously stated constraints on them in the next sixty seconds.
\later
Unsurprisingly, this led to a lot of reader submissions. An awful lot.

You can see the
fallout \href{http://www.reddit.com/r/HPMOR/comments/2xnyi0/113_help_my_evil_plan_has_worked_all_too_well/}{on the /r/HPMOR subreddit}. 
If you’re reading this somewhere that the previous text isn’t a link, you can go to \url{http://www.reddit.com/r/HPMOR} and search for “Help! My evil plan has worked all too well!”
}
%  LocalWords:  Ch

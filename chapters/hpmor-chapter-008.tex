% \chapter{Positive Bias}
\chapter{Positive Voreingenommenheit}

\begin{chapterOpeningAuthorNote}
% All these worlds are J. K. Rowling's, except Europa. Attempt no fanfics there.
% All diese Welten gehören J. K. Rowling, außer Europa. Versucht dort keine Fanfics.
% 
% One alert reviewer asked whether, if Luna is a seer, that means this is going to be an HPDM bottom!Draco mpreg fic. I regret that FFN does not allow me any larger font size in which to say NO. It honestly hadn't occurred to me that Luna might be a \emph{real} seer - I'll have to decide whether to run with that or not - but I think we can all safely assume that if Luna \emph{is} a seer, she said something about “light planting a seed in darkness”, and Xenophilius, as always, interpreted this in rather the wrong way.
Ein aufmerksamer Rezensent fragte, ob, wenn Luna eine Seherin ist, das bedeutet, dass dies ein Draco männlich schwanger fiction wird. Ich bedaure, dass FanFicton.net mir keine größere Schrift erlaubt, in der ich NEIN sagen könnte. Ehrlich gesagt ist mir gar nicht in den Sinn gekommen, dass Luna eine Seherin sein könnte - ich muss mich entscheiden, ob ich damit weitermachen will oder nicht - aber ich denke, wir können alle mit Sicherheit davon ausgehen, dass, wenn Luna eine Seherin ist, sie etwas über 'Licht, das einen Samen in die Dunkelheit pflanzt' gesagt hat, und Xenophilius hat das, wie immer, ziemlich falsch interpretiert.
\end{chapterOpeningAuthorNote}
\begin{chapterOpeningQuote}
% Allow me to warn you that challenging my ingenuity is a dangerous sort of project, and may tend to make your life a lot more surreal.
% Erlaube mir, dich zu warnen, dass es ein gefährliches Unterfangen ist, meine Genialität in Frage zu stellen; es könnte deinen Tag um einiges surrealer machen.
Erlaube mir, dich zu warnen, dass meine Genialität auf die Probe zu stellen, ein gefährliches Vorhaben ist und dein Leben erheblich surrealer machen könnte.
\end{chapterOpeningQuote}

% \lettrine{N}{o}-one had asked for help, that was the problem. They’d just gone around talking, eating, or staring into the air while their parents exchanged gossip. For whatever odd reason, no-one had been sitting down reading a book, which meant she couldn’t just sit down next to them and take out her own book. And even when she’d boldly taken the initiative by sitting down and continuing her third read-through of \emph{Hogwarts: A History,} no-one had seemed inclined to sit down next to her.
\lettrine{N}{iemand} hatte um Hilfe gebeten, das war das Problem. Sie waren nur umhergegangen, hatten geschwatzt, gegessen oder in die Luft gestarrt, während ihre Eltern tratschten. Aus welchem seltsamen Grund auch immer hatte niemand herumgesessen und ein Buch gelesen, was bedeutete, dass sie sich nicht daneben setzen und ihr eigenes Buch herausholen konnte. Und selbst als sie mutig die Initiative ergriffen hatte, indem sie sich selbst hinsetzte und begann, ihre \emph{Geschichte Hogwarts’} zum dritten Mal durchzulesen, hatte sich niemand dazu veranlasst gefühlt, sich neben sie zu setzen.

% Aside from helping people with their homework, or anything else they needed, she really didn’t know how to meet people. She didn’t \emph{feel} like she was a shy person. She thought of herself as a take-charge sort of girl. And yet, somehow, if there wasn’t some request along the lines of “I can’t remember how to do long division” then it was just too \emph{awkward} to go up to someone and say…what? She’d never been able to figure out what. And there didn’t seem to be a standard information sheet, which was ridiculous. The whole business of meeting people had never seemed sensible to her. Why did \emph{she} have to take all the responsibility herself when there were two people involved? Why didn’t adults ever help? She wished some other girl would just walk up to \emph{her} and say, “Hermione, the teacher told me to be friends with you.”
Außer indem sie bei Hausaufgaben oder bei irgendetwas anderem half, wusste sie nicht, wie sie Leute kennenlernen sollte. Nicht, dass sie sich besonders schüchtern \emph{fühlte.} Sie sah sich selbst als jemand, der die Dinge in die Hand nimmt. Und doch – solange es keine Bitte der Art „Ich weiß nicht mehr, wie man schriftlich dividiert“ gab, war es einfach zu \emph{peinlich,} zu Leuten hinzugehen und…\emph{was} zu sagen? Sie hatte nie herausfinden können, was. Und es schien kein Standardinfoblatt zu geben. Das war lächerlich. Das komplette Konzept des Leute-Kennenlernens war ihr nie besonders sinnvoll vorgekommen. Warum musste sie alles tun, wenn zwei Leute daran beteiligt waren? Warum halfen Erwachsene nie? Sie wünschte sich, ein anderes Mädchen würde zu \emph{ihr} kommen und sagen, „Hermine, die Lehrerin sagte, ich solle mich mit dir anfreunden.“

% But let it be quite clear that Hermione Granger, sitting alone on the first day of school in one of the few compartments that had been empty, in the last carriage of the train, with the compartment door left open just in case anyone for any reason wanted to talk to her, was \emph{not} sad, lonely, gloomy, depressed, despairing, or obsessing about her problems. She was, rather, rereading \emph{Hogwarts: A History} for the third time and quite enjoying it, with only a faint tinge of annoyance in the back of her mind at the general unreasonableness of the world.
Aber es muss hier erwähnt werden, dass Hermine Granger, allein an ihrem ersten Schultag in einer der wenigen leeren Kabinen des Zuges sitzend, mit offener Tür, falls jemand aus irgendeinem Grunde mit ihr reden wollen sollte, \emph{nicht} traurig, einsam, deprimiert, verzweifelt oder von ihren Problemen besessen war. Sie war vielmehr dabei, die \emph{Geschichte Hogwarts’} zum dritten Mal zu lesen und hatte durchaus ihre Freude daran, gemischt mit nur einer Spur Genervtheit angesichts der generellen Unvernunft der Welt.

% There was the sound of an inter-train door opening, and then footsteps and an odd slithering sound coming down the hallway of the train. Hermione laid aside \emph{Hogwarts: A History} and stood up and stuck her head outside—just in case someone needed help—and saw a young boy in a wizard’s dress robes, probably first or second year going by his height, and looking quite silly with a scarf wrapped around his head. A small trunk stood on the floor next to him. Even as she saw him, he knocked on the door of another, closed compartment, and he said in a voice only slightly muffled by the scarf, “Excuse me, can I ask a quick question?”
Es ertönte das Geräusch einer sich öffnenden Wagentür, gefolgt von Schritten und einem seltsamen, kriechenden Geräusch, das den Gang entlang kam. Hermine legte die \emph{Geschichte Hogwarts’} beiseite und steckte den Kopf durch die Tür – nur für den Fall, dass jemand Hilfe brauchte – und sah einen Jungen im Zaubererumhang, seiner Größe nach zu urteilen vermutlich im ersten oder zweiten Jahr, der ziemlich albern aussah, da er einen Schal um seinen Kopf gewickelt hatte. Ein kleiner Koffer stand neben ihm auf dem Boden. Als sie ihn sah, klopfte er gerade an die Tür eines anderen Abteils und sagte in einer vom Schal leicht gedämpften Stimme, „Entschuldigt mich, kann ich schnell eine Frage stellen?“

% She didn’t hear the answer from inside the compartment, but after the boy opened the door, she did think she heard him say—unless she’d somehow misheard—“Does anyone here know the six quarks or where I can find a first-year girl named Hermione Granger?”
Sie hörte die Antwort nicht, die aus der Kabine kam, aber nachdem der Junge die Tür geöffnet hatte, dachte sie zu hören – es sei denn, sie hatte ihn missverstanden – „Kann mir einer von euch die Namen der sechs Quarks nennen, oder weiß jemand, wo ich ein Mädchen namens Hermine Granger finden kann?“

% After the boy had closed that compartment door, Hermione said, “Can I help you with something?”
Nachdem der Junge die Kabinentür geschlossen hatte, sagte Hermine, „kann ich dir irgendwie helfen?“

% The scarfed face turned to look at her, and the voice said, “Not unless you can name the six quarks or tell me where to find Hermione Granger.”
Der schalumwickelte Kopf wandte sich ihr zu, und die Stimme sagte: „Nicht, wenn du nicht die Namen der sechs Quarks kennst oder mir sagen kannst, wo ich eine Erstklässlerin namens Hermine Granger finde.“

% “Up, down, strange, charm, truth, beauty, and why are you looking for her?”
„Up, down, strange, charm, truth, beauty – und warum suchst du nach einer Erstklässlerin namens Hermine Granger?“

% It was hard to tell from this distance, but she thought she saw the boy grin widely under his scarf. “Ah, so \emph{you’re} a first-year girl named Hermione Granger,” said that young, muffled voice. “On the train to Hogwarts, no less.” The boy started to walk towards her and her compartment, and his trunk slithered along after him. “Technically, all I needed to do was \emph{look} for you, but it seems likely that I’m meant to talk to you or invite you to join my party or get a key magical item from you or find out that Hogwarts was built over the ruins of an ancient temple or something. PC or NPC, that is the question?”
Es war auf die Distanz schwer zu erkennen, aber sie dachte, den Jungen hinter dem Schal breit grinsen zu sehen. „Ach, \emph{du} bist die Erstklässlerin namens Hermine Granger“, sagte die junge, gedämpfte Stimme. „Und sogar im Zug nach Hogwarts.“ Der Junge fing an, in ihre Richtung zu gehen, und sein Koffer kroch hinter ihm her. „Genau genommen sollte ich nur nach dir suchen, aber es scheint mir naheliegend, dass ich auch mit dir reden soll. Oder dich in meine Abenteuergruppe einladen, oder ein bedeutendes magisches Artefakt von dir erhalten oder herausfinden, dass Hogwarts auf den Ruinen eines alten Tempels erbaut wurde, oder so etwas. PC oder NPC, das ist die Frage.“

% Hermione opened her mouth to reply to this, but then she couldn’t think of any \emph{possible} reply to…\emph{whatever} it was she’d just heard, even as the boy walked over to her, looked inside the compartment, nodded with satisfaction, and sat down on the bench across from her own. His trunk scurried in after him, grew to three times its former diameter and snuggled up next to her own in an oddly disturbing fashion.
Hermine öffnete den Mund um zu antworten, aber ihr fiel keine \emph{geeignete} Antwort auf…was immer das gerade gewesen sein mochte ein, selbst als der Junge zu ihr herüberkam, in das Abteil schaute, zufrieden nickte, und sich auf der leeren Bank gegenüber von ihrem Platz niederließ, wo immer noch das Buch lag. Sein Koffer wuselte hinter ihm her, wuchs auf dreifache Größe, und schmiegte sich in einer beunruhigenden Art und Weise an ihren Koffer.

% “Please, have a seat,” said the boy, “and do please close the door behind you, if you would. Don’t worry, I don’t bite anyone who doesn’t bite me first.” He was already unwinding the scarf from around his head.
„Bitte, setz dich“, sagte der Junge, „und sei so nett, schließ doch die Tür hinter dir. Keine Angst, ich beiße niemanden, der mich nicht zuerst beißt.“ Er war bereits dabei, den Schal um seinen Kopf abzuwickeln.

% The imputation that this boy thought she was \emph{scared} of him made her hand send the door sliding shut, jamming it into the wall with unnecessary force. She spun around and saw a young face with bright, laughing green eyes, and an angry red-dark scar set into his forehead that reminded her of something in the back of her mind but right now she had more important things to think about. “I didn’t say I was Hermione Granger!”
Dass dieser Junge ihr unterstellte, Angst vor ihm zu haben, reichte aus, damit ihre Hand, die Tür mit unnötig viel Kraft zuwarf. Sie wirbelte herum und blickte in ein junges Gesicht mit hellen, lachenden grünen Augen, und einer zornroten Narbe auf seiner Stirn, die sie hintergründig an etwas erinnerte, aber sie hatte in diesem Moment wichtigere Dinge im Kopf. „Ich habe nicht gesagt, dass ich Hermine Granger bin!“

% “\emph{I} didn’t say you \emph{said} you were Hermione Granger, I just said you were Hermione Granger. If you’re asking how I know, it’s because I know everything. Good evening ladies and gentlemen, my name is Harry James Potter-Evans-Verres or Harry Potter for short, I know that probably doesn’t mean anything to \emph{you} for a change—”
„\emph{Ich} habe nicht gesagt, dass du \emph{gesagt hast,} dass du Hermine Granger bist, sondern dass du Hermine Granger \emph{bist.} Falls du dich fragst, woher ich das weiß – es liegt daran, dass ich alles weiß. Guten Abend, meine Damen und Herren, mein Name ist Harry James Potter-Evans-Verres, oder kurz Harry Potter, und ich denke mal, dass \emph{dir} das ausnahmsweise mal gar nichts sagt –“

% Hermione’s mind finally made the connection. The scar on his forehead, the shape of a lightning bolt. “Harry Potter! You’re in \emph{Modern Magical History} and \emph{The Rise and Fall of the Dark Arts} and \emph{Great Wizarding Events of the Twentieth Century.}” It was actually the very first time in her whole life that she’d \emph{met} someone from inside a \emph{book,} and it was a rather odd feeling.
Endlich stellte Hermines Verstand den Zusammenhang her. Die Narbe auf seiner Stirn, in der Form eines Blitzes! „Harry Potter! Du stehst in \emph{Geschichte der modernen Magie} und \emph{Der Aufstieg und Untergang der Dunklen Künste} und \emph{Große Chronik der Zauberei des zwanzigsten Jahrhunderts!}“ Es war das erste Mal in ihrem Leben, dass sie tatsächlich jemandem aus einem Buch begegnete, und es war ein eher seltsames Gefühl.

% The boy blinked three times. “I’m in \emph{books}? Wait, of course I’m in books…what a strange thought.”
Der Junge blinzelte dreimal. „Ich stehe in \emph{Büchern?} Warte, natürlich stehe ich in Büchern…– Was für ein seltsamer Gedanke.“

% “Goodness, didn’t you know?” said Hermione. “I’d have found out everything I could if it was me.”
„Meine Güte, hast du das nicht gewusst?“, sagte Hermine. „Ich jedenfalls hätte alles über mich herausgefunden, wenn ich du gewesen wäre.“

% The boy spoke rather dryly. “Miss~Granger, it has been less than 72~hours since I went to Diagon Alley and discovered my claim to fame. I have spent the last two days buying science books. \emph{Believe me,} I intend to find out everything I can.” The boy hesitated. “What \emph{do} the books say about me?”
Der Junge sprach ziemlich trocken. „Miss~Hermine Granger, es ist weniger als zweiundsiebzig Stunden her, dass ich die Winkelgasse besuchte und von meiner Berühmtheit erfuhr. Ich habe die letzten zwei Tage damit verbracht, Wissenschaftsbücher zu kaufen. \emph{Glaub mir,} ich habe vor, alles herauszufinden, was ich kann.“ Der Junge zögerte. „Was \emph{sagen} denn die Bücher über mich?“

% Hermione Granger’s mind flashed back, she hadn’t realised she would be tested on \emph{those} books so she’d read them only once, but it was just a month ago so the material was still fresh in her mind. “You’re the only one who’s survived the Killing Curse so you’re called the Boy-Who-Lived. You were born to James Potter and Lily Potter formerly Lily Evans on the 31st of July 1980. On the 31st of October 1981 the Dark Lord He-Who-Must-Not-Be-Named though I don’t know why not attacked your home. You were found alive with the scar on your forehead in the ruins of your parents’ house near the burnt remains of You-Know-Who’s body. Chief Warlock Albus Percival Wulfric Brian Dumbledore sent you off somewhere, no-one knows where. \emph{The Rise and Fall of the Dark Arts} claims that you survived because of your mother’s love and that your scar contains all of the Dark Lord’s magical power and that the centaurs fear you, but \emph{Great Wizarding Events of the Twentieth Century} doesn’t mention anything like that and \emph{Modern Magical History} warns that there are lots of crackpot theories about you.”
Hermine versuchte sich zu erinnern. Sie hatte nicht damit gerechnet, über \emph{diese} Bücher ausgefragt zu werden, deswegen hatte sie sie nur einmal gelesen, aber es war erst einen Monat her, deswegen hatte sie den Inhalt noch im Kopf. „Du bist der einzige, der je den Todesfluch überlebte, deswegen nennt man dich den Jungen, der lebt. Du wurdest als Sohn von James Potter und Lily Potter, ehemals Lily Evans, am 31. Juli 1980 geboren. Am 31. Oktober 1981 griff der Dunkle Lord – Er, dessen Name nicht genannt werden darf, auch wenn ich nicht weiß, wieso nicht – euer Haus an, dessen Ort von Sirius Black verraten wurde, auch wenn da nicht drinstand, woher sie wussten, dass er es war. Du wurdest lebend mit der Narbe auf deiner Stirn in den Ruinen deines Elternhauses neben den verbrannten Überresten von ihm, dessen Name nicht genannt werden darf, gefunden. Großmeister Albus Percival Wulfric Brian Dumbledore hat dich irgendwohin gebracht, aber niemand weiß wohin. \emph{Der Aufstieg und Untergang der Dunklen Künste} behauptet, dass du wegen der Liebe deiner Mutter überlebtest, und dass deine Narbe die gesamte Macht des Dunklen Lords enthält, und dass die Zentauren dich fürchten, aber \emph{Große Chronik der Zauberei des zwanzigsten Jahrhunderts} erwähnt nichts dergleichen und \emph{Geschichte der modernen Magie} warnt, dass es jede Menge verrückter Theorien um dich gibt.“

% The boy’s mouth was hanging open. “Were you told to wait for Harry Potter on the train to Hogwarts, or something like that?”
Der Mund des Jungen hing offen. „Wurde dir gesagt, dass du im Zug nach Hogwarts auf Harry Potter warten sollst, oder irgendwas in der Art?“

% “No,” Hermione said. “Who told you about \emph{me?}”
„Nein. Wer hat dir von \emph{mir} erzählt?“

% “Professor McGonagall and I believe I see why. Do you have an eidetic memory, Hermione?”
„Professor McGonagall, und ich glaube, ich verstehe, warum. Hast du ein eidetisches Gedächtnis, Hermine?“

% Hermione shook her head. “It’s not photographic, I’ve always wished it was but I had to read my school books five times over to memorize them all.”
Hermine schüttelte den Kopf. „Es ist nicht fotografisch. Ich habe mir immer gewünscht, dass es das wäre, aber ich musste all meine Schulbücher fünf mal lesen, um sie auswendig zu können.“

% “Really,” the boy said in a slightly strangled voice. “I hope you don’t mind if I test that—it’s not that I don’t believe you, but as the saying goes, ‘Trust, but verify’. No point in wondering when I can just do the experiment.”
„Wirklich“, sagte der Junge mit einer leicht erstickten Stimme. „Ich hoffe, dass es dich nicht stört, wenn ich das teste – nicht, weil ich dir nicht glaube, aber wie das Sprichwort sagt: Vertrauen ist gut, Kontrolle ist besser. Es wäre sinnlos, darüber zu grübeln, wenn ich es einfach per Experiment herausfinden kann.“

% Hermione smiled, rather smugly. She so loved tests. “Go ahead.”
Hermine grinste selbstzufrieden. Sie liebte Tests. „Leg los.“

% The boy stuck a hand into a pouch at his side and said “Magical Drafts and Potions by Arsenius Jigger”. When he withdrew his hand it was holding the book he’d named.
Der Junge steckte eine Hand in seinen Beutel und sagte „Zaubertränke und Zauberbräue von Arsenius Bunsen“. Als er seine Hand wieder hervorzog, hielt er in ihr das genannte Buch.

% Instantly Hermione wanted one of those pouches more than she’d ever wanted anything.
Sofort wünschte sich Hermine so einen Beutel mehr als alles andere.

% The boy opened the book to somewhere in the middle and looked down. “If you were making \emph{oil of sharpness}—”
Der Junge öffnete das Buch irgendwo in der Mitte und blickte hinein. „Wenn du ein \emph{Öl der Schärfe} machen wolltest –“

% “I can \emph{see} that page from here, you know!”
„Ich kann die Seite von hier aus \emph{sehen!}“

% The boy tilted the book so that she couldn’t see it any more, and flipped the pages again. “If you were brewing a \emph{potion of spider climbing,} what would be the next ingredient you added after the Acromantula silk?”
Der Junge kippte das Buch, so dass sie nicht mehr hereinschauen konnte, und blätterte einige Seiten weiter. „Wenn du einen \emph{Trank des Spinnenlaufs} brauen wolltest, was wäre die nächste Zutat, nachdem du die Acromantula-Seide hinzufügtest?“

% “After dropping in the silk, wait until the potion has turned exactly the shade of the cloudless dawn sky, 8 degrees from the horizon and 8 minutes before the tip of the sun first becomes visible. Stir eight times widdershins and once deasil, and then add eight drams of unicorn bogies.”
„Nach dem Hinzugeben der Seide, warte, bis der Trank genau die Farbe des wolkenlosen Morgenhimmels hat, 8 Grad vom Horizont und 8 Minuten bevor die Sonne zu sehen ist. Rühre achtmal gegen den Uhrzeigersinn, einmal im Uhrzeigersinn, und füge dann acht Quäntchen Einhornpopel hinzu.“

% The boy shut the book with a sharp snap and put the book back into his pouch, which swallowed it with a small burping noise. “Well well well \emph{well} well well. I should like to make you a proposition, Miss~Granger.”
Der Junge schloss das Buch mit einem Klatschen, und tat es zurück in seinen Beutel, welcher es mit einem kleinen Rülpslaut verschlang. „Sehr sehr \emph{sehr} sehr gut. Ich möchte dir hiermit ein Angebot machen, Hermine Granger.“

% “A proposition?” Hermione said suspiciously. Girls weren’t supposed to listen to those.
„Ein Angebot?“, sagte Hermine misstrauisch. Mädchen sollten auf so etwas nicht hören.

% It was also at this point that Hermione realised the other thing—well, one of the things—which was odd about the boy. Apparently people who were \emph{in} books actually \emph{sounded} like a book when they talked. This was quite the surprising discovery.
In diesem Moment bemerkte Hermine auch die andere Sache – naja, eine andere Sache – die an dem Jungen ungewöhnlich war. Augenscheinlich war es nämlich so, dass Leute, die \emph{in} Büchern waren, \emph{wie} Bücher klangen, wenn sie sprachen. Es war eine ziemlich überraschende Feststellung.

% The boy reached into his pouch and said, “can of pop”, retrieving a bright green cylinder. He held it out to her and said, “Can I offer you something to drink?”
Der Junge griff in seinen Beutel, sagte „Limonadendose“ und zog einen hellgrünen Zylinder hervor. Er hielt ihn hin und sagte, „Kann ich dir etwas zu trinken anbieten?“

% Hermione politely accepted the fizzy drink. In fact she \emph{was} feeling sort of thirsty by now. “Thank you very much,” Hermione said as she popped the top. “Was that your proposition?”
Hermine nahm die Dose höflich an. In der Tat \emph{war} sie gerade etwas durstig. „Vielen Dank“, sagte Hermine, als sie den Deckel öffnete. „War das dein Angebot?“

% The boy coughed. “No,” he said. Just as Hermione started to drink, he said, “I’d like you to help me take over the universe.”
Der Junge hustete. „Nein“, sagte er. Gerade als Hermine zu trinken begann, sagte er, „Ich möchte, dass du mir bei der Eroberung des Universums hilfst.“

% Hermione finished her drink and lowered the can. “No thank you, I’m not evil.”
Hermine trank zu Ende und senkte die Limo. „Nein danke, aber ich bin nicht böse.“

% The boy looked at her in surprise, as though he’d been expecting some other answer. “Well, I was speaking a bit rhetorically,” he said. “In the sense of the Baconian project, you know, not political power. ‘The effecting of all things possible’ and so on. I want to conduct experimental studies of spells, figure out the underlying laws, bring magic into the domain of science, merge the wizarding and Muggle worlds, raise the entire planet’s standard of living, move humanity centuries ahead, discover the secret of immortality, colonize the Solar System, explore the galaxy, and most importantly, figure out what the heck is really going on here because all of this is blatantly impossible.”
Der Junge schaute sie überrascht an, so, als hätte er eine andere Antwort erwartet. „Naja, ich habe etwas rhetorisch gesprochen“, sagte er. „Im Sinne des Baconschen Projekts, weißt du, nicht politische Macht. ’Die Bewerkstelligung aller möglichen Dinge‘ und so weiter. Ich möchte experimentelle Studien an Zaubersprüchen durchführen, die zugrundeliegenden Gesetze herausfinden, die Magie in den Bereich der Wissenschaft bringen, die Zauberer- und Muggelwelt zusammenführen, den Lebensstandard aller auf diesem Planeten heben, die Menschheit hunderte von Jahren voranbringen, das Geheimnis der Unsterblichkeit entdecken, das Sonnensystem bevölkern, die Galaxie erforschen, und wichtiger noch, herausfinden, was zum Teufel hier eigentlich los ist, denn das alles hier ist einfach nur unmöglich.“

% That sounded a bit more interesting. “And?”
Das klang etwas interessanter. „Und?“

% The boy stared at her incredulously. “\emph{And?} That’s not \emph{enough?}”
Der Junge schaute sie ungläubig an. „\emph{Und?} Das ist \emph{nicht genug?}“

% “And what do you want from me?” said Hermione.
„Und was möchtest du von mir?“ sagte Hermine.

% “I want you to help me do the research, of course. With your encyclopedic memory added to my intelligence and rationality, we’ll have the Baconian project finished in no time, where by ‘no time’ I mean probably at least thirty-five years.”
„Ich möchte natürlich, dass du mir bei meinen Forschungen hilfst. Mit deinem enzyklopädischen Gedächtnis, zusammen mit meiner Intelligenz und meiner Rationalität werden wir das Baconsche Projekt in kürzester Zeit beendet haben, wobei ich mit 'kürzester Zeit' wahrscheinlich 35 Jahre oder mehr meine.“

% Hermione was beginning to find this boy annoying. “I haven’t seen you do anything intelligent. Maybe I’ll let \emph{you} help me with \emph{my} research.”
Hermine begann, diesen Jungen nervig zu finden. „Ich habe dich noch nichts Intelligentes machen sehen. Vielleicht erlaube ich \emph{dir,} mir bei \emph{meinen} Studien zu helfen.“

% There was a certain silence in the compartment.
Eine bestimmte Art von Stille füllte das Abteil.

% “So you’re asking me to demonstrate my intelligence, then,” said the boy after a long pause.
„Du möchtest also eine Demonstration meiner Intelligenz“, sagte der Junge nach einer langen Pause.

% Hermione nodded.
Hermine nickte.

% “I warn you that challenging my ingenuity is a dangerous project, and tends to make your life a lot more surreal.”
„Erlaube mir, dich zu warnen, dass es ein gefährliches Unterfangen ist, meine Genialität in Frage zu stellen; es könnte deinen Tag um einiges surrealer machen.“

% “I’m not impressed yet,” Hermione said. Unnoticed, the green drink once again rose to her lips.
„Bisher bin ich nicht beeindruckt“, sagte Hermine. Ihre Hand mit der Limonadendose darin hob sich wieder zu ihrem Mund.

% “Well, maybe \emph{this} will impress you,” the boy said. He leaned forward and looked at her intensely. “I’ve already done a bit of experimenting and I found out that I don’t need the wand, I can make anything I want happen just by snapping my fingers.”
„Na, dann wird \emph{das} dich vielleicht beeindrucken“, sagte der Junge. Er lehnte sich vor und schaute sie durchdringend an. „Ich habe bereits ein wenig experimentiert, und herausgefunden, dass ich meinen Zauberstab nicht brauche. Ich kann alles, was ich will, einfach geschehen lassen, indem ich mit den Fingern schnippe.“

% It came just as Hermione was in the middle of swallowing, and she choked and coughed and expelled the bright green fluid.
Das sagte er genau in dem Moment, als Hermine gerade schluckte, und sie verschluckte sich und hustete und spuckte die hellgrüne Flüssigkeit wieder aus.

% Onto her brand new, never-before-worn witch’s robes, on the very first day of school.
Auf ihren nagelneuen, noch nie getragenen Zaubererumhang, am allerersten Schultag.

% Hermione actually screamed. It was a high-pitched sound that sounded like an air raid siren in the closed compartment. “\emph{Eek! My clothes!}”
Hermine schrie auf. Es war ein hoher, durchdringender Ton, der in dem geschlossenen Abteil wie ein Fliegeralarm klang. „\emph{Ihks! Meine Kleidung!}“

% “Don’t panic!” said the boy. “I can fix it for you. Just watch!” He raised a hand and snapped his fingers.
„Keine Panik!“, sagte der Junge. „Ich kann es wieder gut machen. Schau hin!“ Er hob seine Hand und schnippte mit den Fingern.

% “You’ll—” Then she looked down at herself.
„Du –“ Dann schaute sie an sich runter.

% The green fluid was still there, but even as she watched, it started to vanish and fade and within just a few moments, it was like she’d never spilled anything at herself.
Die grüne Flüssigkeit war noch da, aber sie verschwand beim Hinschauen, und innerhalb weniger Momente war es, als hätte sie nie die Limo auf dem Umhang verteilt.

% Hermione stared at the boy, who was wearing a rather smug sort of smile.
Hermine starrte den Jungen an, der ein selbstgefälliges Lächeln aufgesetzt hatte.

% Wordless wandless magic! At \emph{his} age? When he’d only got his schoolbooks \emph{three days} ago?
Wortlose stablose Magie! In \emph{seinem} Alter! Wo er doch seine Bücher erst vor \emph{drei Tagen} erhalten hatte?

% Then she remembered what she’d read, and she gasped and flinched back from him. \emph{All the Dark Lord’s magical power! In his scar!}
Dann erinnerte sie sich an das, was sie gelesen hatte, zog den Atem ein und zuckte von ihm weg. \emph{Die gesamte Macht des dunklen Lords! In seiner Narbe!}

% She rose hastily to her feet. “I, I, I need to go the toilet, wait here all right—” she had to find a grown-up she had to tell them—
Hastig stand sie auf. „Ich, ich, ich muss auf die Toilette, warte hier, okay? –“ Sie musste einen Erwachsenen finden, sie musste es jemandem sagen –

% The boy’s smile faded. “It was just a trick, Hermione. I’m sorry, I didn’t mean to scare you.”
Das Lächeln des Jungen verschwand. „Es war nur ein Trick, Hermine. Es tut mir leid, ich hatte nicht vor, dich zu verängstigen.“

% Her hand halted on the door handle. “A \emph{trick?}”
Ihre Hand hielt am Türgriff inne. „Ein \emph{Trick?}“

% “Yes,” said the boy. “You asked me to demonstrate my intelligence. So I did something apparently impossible, which is always a good way to show off. I can’t \emph{really} do anything just by snapping my fingers.” The boy paused. “At least I don’t \emph{think} I can, I’ve never actually tested it experimentally.” The boy raised his hand and snapped his fingers again. “Nope, no banana.”
„Ja“, sagte der Junge. „Du hast mich gebeten, meine Intelligenz zu demonstrieren. Ich tat also etwas augenscheinlich Unmögliches – immer eine gute Art anzugeben. Ich kann nicht \emph{wirklich} alles durch ein Fingerschnippen geschehen lassen.“ Der Junge machte eine Pause. „Zumindest \emph{denke} ich nicht, dass ich das kann, ich habe es nie probiert.“ Der Junge hob seine Hand und schnippte noch einmal mit den Fingern. „Nein, keine Banane.“

% Hermione was as confused as she’d ever been in her life.
Hermine war so verwirrt, wie sie es in ihrem ganzen Leben noch nicht gewesen war.

% The boy was now smiling again at the look on her face. “I did \emph{warn} you that challenging my ingenuity tends to make your life surreal. Do remember this the next time I warn you about something.”
Der Junge lächelte nun wieder über den Ausdruck in ihrem Gesicht. „Ich hatte dich \emph{gewarnt,} dass es ein gefährliches Unterfangen ist, meine Genialität in Frage zu stellen. Denke daran, wenn ich dich das nächste Mal vor etwas warne.“

% “But, but,” Hermione stammered. “What did you \emph{do,} then?”
„Aber, aber“, stotterte Hermine, „was hast du denn \emph{dann} getan?“

% The boy’s gaze took on a measuring, weighing quality that she’d never seen before from someone her own age. “You think you have what it takes to be a scientist in your own right, with or without my help? Then let’s see how \emph{you} investigate a confusing phenomenon.”
Der Blick des Jungen nahm einen einschätzenden und bewertenden Ausdruck an, den sie noch nie von jemandem in ihrem Alter gesehen hatte. „Du denkst, dass du es in dir hast, eine Wissenschaftlerin zu werden, egal, ob ich dir helfe oder nicht? Dann zeig mal, wie \emph{du} ein verwirrendes Phänomen untersuchst.“

% “I…” Hermione’s mind went blank for a moment. She loved tests but she’d never had a test like \emph{this} before. Frantically, she tried to cast back for anything she’d read about what scientists were supposed to do. Her mind skipped gears, ground against itself, and spat back the instructions for doing a science investigation project:
„Ich…“ Hermines Gehirn setzte für einen Moment aus. Sie liebte es, getestet zu werden, aber \emph{so} eine Aufgabe hatte sie noch nie gestellt bekommen. Hastig durchstöberte sie ihre Erinnerungen nach allen Dingen, die ein Wissenschaftler tun sollte. Ihr Gehirn schaltete sich ein, fing an, hart zu arbeiten, und spuckte schließlich die Anweisungen zur Durchführung eines „Jugend forscht“-Projekts aus:

\emph{%
% Step 1: Form a hypothesis.\\
Schritt 1: Stelle eine Hypothese auf.\\
% Step 2: Do an experiment to test your hypothesis.\\
Schritt 2: Führe ein Experiment durch, um deine Hypothese zu testen.\\
% Step 3: Measure the results.\\
Schritt 3: Miss das Resultat.\\
% Step 4: Make a cardboard poster.%
Schritt 4: Stelle dein Ergebnis auf einem Poster dar.%
}

% Step 1 was to form a hypothesis. That meant, try to think of something that \emph{could} have happened just now. “All right. My hypothesis is that you cast a Charm on my robes to make anything spilled on it vanish.”
Der erste Schritt war, eine Hypothese aufzustellen. Das bedeutete, sich auszudenken, was gerade passiert sein \emph{könnte.} „Also gut. Meine Hypothese ist es, dass du einen Zauberspruch auf meinen Umhang gesprochen hast, der die darauf verschüttete Limonade verschwinden ließ.“

% “All right,” said the boy, “is that your answer?”
„Okay“, sagte der Junge, „ist das deine Antwort?“

% The shock was wearing off, and Hermione’s mind was starting to work properly. “Wait, that can’t be right. I didn’t see you touch your wand or say any spells so how could you have cast a Charm?”
Der Schock ließ nach, und Hermines Gedanken fingen an, richtig zu funktionieren. „Warte, das ist keine gute Idee. Ich habe dich weder deinen Zauberstab berühren sehen, noch irgendwelche Worte sprechen hören, wie könntest du also einen Zauberspruch gesprochen haben?“

% The boy waited, his face neutral.
Der Junge wartete mit neutralem Gesichtsausdruck.

% “But suppose all the robes come from the store with a Charm \emph{already} on them to keep them clean, which would be a useful sort of Charm for them to have. You found that out by spilling something on \emph{yourself} earlier.”
„Aber nehmen wir mal an, die Umhänge werden bereits mit einem Selbstreinigungszauber verkauft, was ein sehr nützlicher Zauber für sie wäre. Das hast du vorher herausgefunden, als du \emph{selber} etwas verkleckert hast.“

% Now the boy’s eyebrows lifted. “Is \emph{that} your answer?”
Nun hoben sich die Augenbrauen des Jungen. „Ist \emph{das} deine Antwort?“

% “No, I haven’t done Step 2, ‘Do an experiment to test your hypothesis.’”
„Nein, ich habe noch nicht Schritt 2 durchgeführt, 'Führe ein Experiment durch, um deine Hypothese zu testen.“

% The boy closed his mouth again, and began to smile.
Der Junge schloss den Mund wieder, und begann zu lächeln.

% Hermione looked at the drinks can, which she’d automatically put into the cup-holder at the window. She took it up and peered inside, and found that it was around one-third full.
Hermine schaute die Limonadendose in ihrer Hand an, welche sie wie automatisch im Tassenhalter am Fenster abgestellt hatte. Sie betrachtete sie, und stellte fest, dass sie etwa zu einem Drittel voll war.

% “Well,” said Hermione, “the experiment I want to do is to pour it on my robes and see what happens, and my prediction is that the stain will disappear. Only if it \emph{doesn’t} work, my robes will be stained, and I don’t want that.”
„Nun gut“, sagte Hermine, „das Experiment, was ich durchführen will, ist, die Limo auf meinen Umhang zu gießen und zu sehen, was passiert, und meine Vorhersage ist, dass sie verschwinden wird. Nur – wenn es \emph{nicht} klappt, beschmutze ich mir meinen Umhang, und das will ich nicht.“

% “Do it to mine,” said the boy, “that way you don’t have to worry about your robes getting stained.”
„Tu es mit meinem“, sagte der Junge, „dann musst du dir keine Sorgen machen, dass deiner schmutzig wird.“

% “But—” Hermione said. There was something \emph{wrong} with that thinking but she didn’t know how to say it exactly.
„Aber –“, sagte Hermine. Irgendwas war an dieser Art zu denken \emph{falsch,} aber sie konnte nicht genau sagen was.

% “I have spare robes in my trunk,” said the boy.
„Ich habe weitere Umhänge in meinem Koffer“, sagte der Junge.

% “But there’s nowhere for you to change,” Hermione objected. Then she thought better of it. “Though I suppose I could leave and close the door—”
„Aber du hast hier keinen Platz, wo du dich umzuziehen kannst“, wandte Hermine ein. Dann dachte sie nochmal drüber nach. „Wobei ich denke, dass ich so lange herausgehen und die Tür schließen könnte –“

% “I have somewhere to change in my trunk, too.”
„Ich habe in dem Koffer auch Platz zum Umziehen.“

% Hermione looked at his trunk, which, she was beginning to suspect, was rather more special than her own.
Hermine schaute seinen Koffer an, und begann zu vermuten, dass er irgendwie eine ganze Ecke spezieller als ihrer war.

% “All right,” Hermione said, “since you say so,” and she rather gingerly poured a bit of green pop onto a corner of the boy’s robes. Then she stared at it, trying to remember how long the original fluid had taken to disappear…
„Also gut“, sagte Hermine, „wenn du das sagst“, und sie kippte zögerlich ein bisschen grüne Limonade auf eine Ecke des Umhangs des Jungen. Dann starrte sie drauf und versuchte sich zu erinnern, wie lange es bei ihr gedauert hatte, bis die Limonade verschwunden war…

% And the green stain vanished!
Und die Limonade verschwand!

% Hermione let out a sigh of relief, not least because this meant she wasn’t dealing with all of the Dark Lord’s magical power.
Hermine atmete auf, unter anderem, weil dies bedeutete, dass sie es nicht mit der gesamten Macht des Dunklen Lords zu tun hatte.

% Well, Step 3 was measuring the results, but in this case that was just seeing that the stain had vanished. And she supposed she could probably skip Step 4, about the cardboard poster. “My answer is that the robes are Charmed to keep themselves clean.”
Nun, Schritt drei war, das Resultat zu messen, aber das bestand nur darin, zu sehen, dass die Limonade verschwand. Und sie dachte sich, dass sie Schritt 4 mit dem Plakat auch überspringen konnte. „Meine Antwort ist, dass die Umhänge verzaubert sind, um sauber zu bleiben.“

% “Not quite,” said the boy.
„Nicht ganz“, sagte der Junge.

% Hermione felt a stab of disappointment. She really wished she \emph{wouldn’t} have felt that way, the boy wasn’t a teacher, but it was still a test and she’d got a question wrong and that always felt like a little punch in the stomach.
Hermine fühlte einen Stich der Enttäuschung. Sie wünschte, dass sie sich nicht so fühlen würde, der Junge war zwar kein Lehrer, aber es war dennoch ein Test und sie hatte die falsche Antwort gegeben und das fühlte sich immer wie ein leichter Schlag in den Magen an.

% (It said almost everything you needed to know about Hermione Granger that she had never let that stop her, or even let it interfere with her love of being tested.)
(Eigentlich sagte es alles über Hermine, was man wissen musste; dass sie sich nämlich niemals von so etwas aufhalten ließ, geschweige denn ihre Lust, geprüft zu werden, verlor.)

% “The sad thing is,” said the boy, “you probably did everything the book told you to. You made a prediction that would distinguish between the robe being charmed and not charmed, and you tested it, and rejected the null hypothesis that the robe was not charmed. But unless you read the very, very best sort of books, they won’t quite teach you how to do science \emph{properly}. Well enough to \emph{really} get the right answer, I mean, and not just churn out another publication like Dad always complains about. So let me try to explain—without giving away the answer—what you did wrong this time, and I’ll give you another chance.”
„Das Traurige ist“, sagte der Junge, „dass du vermutlich alles getan hast, wie es im Buch steht. Du hast eine Vorhersage aufgestellt, die zwischen einem verzauberten und einem unverzauberten Umhang unterscheiden würde, du hast sie getestet, und die Nullhypothese aufgegeben, dass der Umhang nicht verzaubert ist. Aber solange du nicht die aller-allerbesten Bücher liest, wirst du nicht \emph{wirklich} lernen, wie man Wissenschaft betreibt. Ich meine, gut genug um \emph{wirklich} auf die Antwort zu kommen, und nicht einfach nur ein Paper nach dem anderen herauszubringen, wie die über die sich Papa immer beschwert. Also lass mich – ohne die Antwort zu verraten – erklären, was du falsch gemacht hast, und ich gebe dir eine neue Chance.“

% She was starting to resent the boy’s oh-so-superior tone when he was just another eleven-year-old like her, but that was secondary to finding out what she’d done wrong. “All right.”
Sie hatte begonnen, dem Jungen seinen ach-so-überlegenen Tonfall übelzunehmen, wo er doch genau wie sie erst elf Jahre alt war, doch gegenüber ihrem Drang, herauszufinden, was sie falsch gemacht hatte, war das nachrangig. „Okay, meinetwegen.“

% The boy’s expression grew more intense. “This is a game based on a famous experiment called the 2-4-6 task, and this is how it works. I have a \emph{rule}—known to me, but not to you—which fits some triplets of three numbers, but not others. 2-4-6 is one example of a triplet which fits the rule. In fact…let me write down the rule, just so you know it’s a fixed rule, and fold it up and give it to you. Please don’t look, since I infer from earlier that you can read upside-down.”
Der Gesichtsausdruck des Jungen wurde intensiver. „Dies ist ein Spiel, das auf einem berühmten Experiment, der sogenannten 2-4-6-Aufgabe, basiert, und es geht so: Ich habe eine \emph{Regel} – die mir, aber nicht dir bekannt ist – welche auf einige Zahlentripel zutrifft, aber auf andere nicht. 2-4-6 ist ein Beispiel für ein Tripel, das der Regel entspricht. Lass mich schnell die Regel aufschreiben, damit du weißt, dass sie festgelegt ist, den Zettel zusammenfalten und dir geben. Schau bitte nicht zu, ich hab eben festgestellt, dass du auf dem Kopf lesen kannst.“

% The boy said “paper” and “mechanical pencil” to his pouch, and she shut her eyes tightly while he wrote.
Der Junge sagte „Papier“ und „Druckbleistift“ zu seinem Beutel, und sie schloss ihre Augen fest, als er schrieb.

% “There,” said the boy, and he was holding a tightly folded piece of paper. “Put this in your pocket,” and she did.
„So“, sagte der Junge, und er hielt ein mehrfach gefaltetes Blatt Papier in der Hand. „Steck das bitte in deine Tasche“, und sie tat genau das.

% “Now the way this game works,” said the boy, “is that you give me a triplet of three numbers, and I’ll tell you ‘Yes’ if the three numbers are an instance of the rule, and ‘No’ if they’re not. I am Nature, the rule is one of my laws, and you are investigating me. You already know that 2-4-6 gets a ‘Yes’. When you’ve performed all the further experimental tests you want—asked me as many triplets as you feel necessary—you stop and guess the rule, and then you can unfold the sheet of paper and see how you did. Do you understand the game?”
„Das Spiel funktioniert so“, sagte der Junge, „dass du mir ein Zahlentripel sagst, und ich sage dir 'Ja', wenn die drei Zahlen ein Beispiel für die Regel sind, und 'Nein', wenn nicht. Du weißt bereits, dass 2-4-6 mit 'ja' beantwortet wird. Wenn du alle Tests durchgeführt hast, die du machen willst – also nach so vielen Tripeln, wie du für nötig hältst, gefragt hast, hörst du auf und rätst die Regel, und dann kannst du das Papier auffalten und sehen, ob du Recht hattest. Verstehst du das Spiel?“

% “Of course I do,” said Hermione.
„Natürlich tue ich das“, sagte Hermine.

% “Go.”
„Dann leg los.“

% “4–6–8” said Hermione.
„4-6-8“, sagte Hermine.

% “Yes,” said the boy.
„Ja“, sagte der Junge.

% “10–12–14”, said Hermione.
„10-12-14“, sagte Hermine.

% “Yes,” said the boy.
„Ja“, sagte der Junge.

% Hermione tried to cast her mind a little further afield, since it seemed like she’d already done all the testing she needed, and yet it couldn’t be that easy, could it?
Hermine versuchte, mit ihren Gedanken etwas weiter auszuholen, da es ihr schien, dass sie bereits alle nötigen Tests durchgeführt hatte. Aber es konnte ja kaum so einfach sein, oder?

% “1–3–5.”
„1-3-5.“

% “Yes.”
„Ja“

% “Minus 3, minus 1, plus 1.”
„Minus 3, minus 1, plus 1.“

% “Yes.”
„Ja.“

% Hermione couldn’t think of anything else to do. “The rule is that the numbers have to increase by two each time.”
Hermine konnte sich nichts anderes mehr vorstellen. „Die Regel ist, dass die Zahlen jedes mal um zwei hochgehen müssen.“

% “Now suppose I tell you,” said the boy, “that this test is harder than it looks, and that only 20\% of grown-ups get it right.”
„Lass mich erwähnen“, sagte der Junge, „dass dieser Test schwerer ist, als er aussieht, und dass nur 20\% aller Erwachsenen ihn bestehen.“

% Hermione frowned. What had she missed? Then, suddenly, she thought of a test she still needed to do.
Hermine runzelte die Stirn. Was hatte sie übersehen? Dann fiel ihr schlagartig ein Test ein, den sie noch tun musste.

% “2–5–8!” she said triumphantly.
„2-5-8!“, sagte sie triumphierend.

% “Yes.”
„Ja.“

% “10–20–30!”
„10-20-30!“

% “Yes.”
„Ja.“

% “The real answer is that the numbers have to go up by the \emph{same} amount each time. It doesn’t have to be 2.”
„Die richtige Antwort ist, dass die Zahlen jedes Mal um den \emph{selben Betrag} steigen müssen, es muss nicht immer 2 sein.“

% “Very well,” said the boy, “take the paper out and see how you did.”
„Sehr gut“, sagte der Junge. „Nimm den Zettel und schau, wie du dich geschlagen hast.“

% Hermione took the paper out of her pocket and unfolded it.
Hermine nahm den Zettel aus ihrer Hosentasche und entfaltete ihn.

% \emph{Three real numbers in increasing order, lowest to highest.}
\emph{Drei reelle Zahlen in steigender Ordnung, mit der niedrigsten zuerst.}

% Hermione’s jaw dropped. She had the distinct feeling of something terribly unfair having been done to her, that the boy was a dirty rotten cheating liar, but when she cast her mind back she couldn’t think of any wrong responses that he’d given.
Hermines Kinn klappte herunter. Sie hatte das dumpfe Gefühl, dass ihr etwas schrecklich Unfaires angetan wurde, dass der Junge ein dreckiger Betrüger und Lügner war, aber wenn sie darüber nachdachte, konnte sie keine falschen Antworten finden, die er gegeben hatte.

% “What you’ve just discovered is called ‘positive bias’,” said the boy. “You had a rule in your mind, and you kept on thinking of triplets that should make the rule say ‘Yes’. But you didn’t try to test any triplets that should make the rule say ‘No’. In fact you didn’t get a \emph{single} ‘No’, so ‘any three numbers’ could have just as easily been the rule. It’s sort of like how people imagine experiments that could confirm their hypotheses instead of trying to imagine experiments that could falsify them—that’s not quite exactly the same mistake but it’s close. You have to learn to look on the negative side of things, stare into the darkness. When this experiment is performed, only 20\% of grown-ups get the answer right. And many of the others invent fantastically complicated hypotheses and put great confidence in their wrong answers since they’ve done so many experiments and everything came out like they expected.”
„Was du gerade entdeckt hast, nennt sich in der Literatur \emph{Positive Voreingenommenheit}“, sagte der Junge. „Du hattest eine Regel im Kopf und du hast nur an Zahlentripel gedacht, die die Regel dazu gebracht haben, ’Ja‘ zu sagen. Aber du hast nicht probiert, so viele Tripel wie möglich zu finden, die die Regel 'Nein' sagen zu lassen. In der Tat hast du \emph{kein einziges} ’Nein‘ bekommen, deswegen hätte die Regel genauso gut ’drei beliebige Zahlen‘ lauten können. Es ist so ähnlich wie die Leute, die sich Experimente ausdenken, die ihre Hypothesen bestätigen könnten, anstatt solche zu finden, welche sie widerlegen könnten – nicht ganz derselbe Fehler, aber fast. Du musst lernen, die negative Seite zu sehen, in die Finsternis zu schauen. Wenn man dieses Experiment durchführt, bekommen es nur 20\% der Erwachsenen hin. Und viele der anderen denken sich faszinierend komplizierte Hypothesen aus und haben großes Vertrauen in ihre falschen Antworten, da sie so viele Experimente durchgeführt haben und alles passiert ist, wie sie es erwartet haben.“

% “Now,” said the boy, “do you want to take another shot at the original problem?”
„Also“, sagte der Junge, „möchtest du dich ein weiteres Mal an dem ursprünglichen Problem probieren?“

% His eyes were quite intent now, as though this were the \emph{real} test.
Seine Augen sahen sehr fokussiert aus, als wäre dies der \emph{echte} Test.

% Hermione shut her eyes and tried to concentrate. She was sweating underneath her robes. She had an odd feeling that this was the hardest she’d ever been asked to think on a test or maybe even the \emph{first} time she’d ever been asked to think on a test.
Hermine schloss ihre Augen und probierte sich zu konzentrieren. Sie schwitzte unter ihrem Umhang. Sie hatte das komische Gefühl, dass sie nie zuvor bei einem Test so intensiv nachgedacht hatte, oder vielleicht sogar zum \emph{ersten} Mal bei einem Test nachdenken musste.

% What other experiment could she do? She had a Chocolate Frog, could she try to rub some of that on the robes and see if \emph{it} vanished? But that still didn’t seem like the kind of twisty negative thinking the boy was asking for. Like she was still asking for a ‘Yes’ if the Chocolate Frog stain disappeared, rather than asking for a ‘No’.
Was für ein anderes Experiment konnte sie durchführen? Sie hatte einen Schokoladenfrosch, könnte sie davon etwas auf ihren Umhang reiben und schauen, ob \emph{das} verschwand? Aber das schien ihr noch nicht das verdrehte negative Denken zu sein, von dem der Junge geredet hatte. Als würde sie immer noch nach einem ’Ja‘ fragen, wenn der Schokoladenfleck verschwand, als nach einem ’Nein‘ zu fragen.

% So…on her hypothesis…when should the pop…\emph{not} vanish?
Also…ihrer Hypothese nach…wann sollte die Limonade…\emph{nicht} verschwinden?

% “I have an experiment to do,” Hermione said. “I want to pour some pop on the floor, and see if it \emph{doesn’t} vanish. Do you have some paper towels in your pouch, so I can mop up the spill if this doesn’t work?”
„Ich habe mir ein Experiment überlegt“, sagte Hermine. „Ich möchte Limonade auf den Boden schütten, und schauen, ob sie \emph{nicht} verschwindet. Hast du Taschentücher in deiner Tasche, sodass ich die Limonade aufwischen kann, falls es nicht klappt?“

% “I have napkins,” said the boy. His face still looked neutral.
„Ich habe Servietten“, sagte der Junge. Sein Gesicht sah immer noch neutral aus.

% Hermione took the can, and poured a small bit of pop onto the floor.
Hermine nahm die Limonadendose, und schüttete etwas Limonade auf den Boden.

% A few seconds later, it vanished.
Ein paar Sekunden später verschwand sie.

% Then the realisation hit her and she felt like kicking herself. “Of course! \emph{You} gave me that can! It’s not the robe that’s enchanted, it was the pop all along!”
„Heureka“, sagte Hermine leise. Es war wie ein Zwang, sie \emph{musste} es einfach sagen. In der Tat fühlte sie sich danach, es zu schreien, aber dafür war sie doch etwas zu gehemmt. Dann wurde ihr klar, was passiert war, und sie hätte sich selbst ohrfeigen können. „Natürlich! \emph{Du} hast mir die Limonade gegeben! Es war nicht der Umhang, der verzaubert war, es war von vorneherein die Limonade!“

% The boy stood up and bowed to her solemnly. He was grinning widely now. “Then…may I help you with your research, Hermione Granger?”
Der Junge stand auf und verbeugte sich feierlich vor ihr. Er grinste nun sehr weit. „Also…darf ich dir bei deinen Studien behilflich sein, Hermine Granger?“

% “I, ah…” Hermione was still feeling the rush of euphoria, but she wasn’t quite sure about how to answer \emph{that.}
„Ich, ähm…“ Hermine spürte immer noch die Euphorie in sich wallen, aber sie wusste nicht, was sie \emph{darauf} antworten sollte.

% They were interrupted by a weak, tentative, faint, rather \emph{reluctant} knocking at the door.
Sie wurden von einem schwachen, zögernden, eher \emph{widerstrebenden} Klopfen an der Tür unterbrochen.

% The boy turned and looked out the window, and said, “I’m not wearing my scarf, so can you get that?”
Der Junge drehte sich um und schaute aus dem Fenster, und sagte, „Ich trage meinen Schal nicht, kannst du bitte öffnen?“

% It was at this point that Hermione realised why the boy—no, the Boy-Who-Lived, Harry Potter—had been wearing the scarf over his head in the first place, and felt a little silly for not realising it earlier. It was actually sort of odd, since she would have thought Harry Potter would proudly display himself to the world; and the thought occurred to her that he might actually be shyer than he seemed.
In diesem Moment verstand Hermine, warum der Junge – nein, der Junge, der lebt, Harry Potter – die ganze Zeit den Schal um sein Gesicht gewickelt gehabt hatte, und fühlte sich etwas dumm, weil sie es nicht früher verstanden hatte. Es war etwas komisch, da sie gedacht hätte, Harry Potter müsse die Art von Junge sein, der sich stolz der Welt stellte; und ihr kam der Gedanke, dass er schüchterner sein könnte als er wirkte.

% When Hermione pulled the door open, she was greeted by a trembling young boy who looked exactly like he knocked.
Als Hermine die Tür aufzog, stand ihr ein zitternder Junge gegenüber, der genauso aussah, wie er geklopft hatte.

% “Excuse me,” said the boy in a tiny voice, “I’m Neville Longbottom. I’m looking for my pet toad, I, I can’t seem to find it anywhere on this carriage…have you seen my toad?”
„Entschuldigt bitte“, sagte er in einer winzigen Stimme. „Ich bin Neville Longbottom. Ich suche nach meiner Kröte, ich, ich kann sie nirgendwo in diesem Wagen finden…Habt ihr meine Kröte gesehen?“

% “No,” Hermione said, and then her helpfulness kicked in full throttle. “Have you checked all the other compartments?”
„Nein“, sagte Hermine und dann setzte ihre Hilfsbereitschaft mit voller Kraft ein. „Hast du alle anderen Kabinen durchsucht?“

% “Yes,” whispered the boy.
„Ja“, flüsterte der Junge.

% “Then we’ll just have to check all the other carriages,” Hermione said briskly. “I’ll help you. My name is Hermione Granger, by the way.”
„Dann werden wir wohl die anderen Wagen durchsuchen müssen“, sagte Hermine schnell. „Ich werde dir helfen. Mein Name ist übrigens Hermine Granger.“

% The boy looked like he might faint with gratitude.
Der Junge schaute sie dankbar an.

% “Hold on,” came the voice of the \emph{other} boy—Harry Potter. “I’m not sure that’s the best way to do it.”
„Wartet mal“, kam die Stimme vom \emph{anderen} Jungen – Harry Potter. „Ich glaube nicht, dass das die beste Art ist, es zu tun.“

% At this Neville looked like he might cry, and Hermione swung around, angered. If Harry Potter was the sort of person who’d abandon a little boy just because he didn’t want to be interrupted…“What? Why \emph{not?}”
Darauf sah Neville aus, als würde er gleich weinen, und Hermine drehte sich wutentbrannt um. Wenn Harry Potter die Art von Junge war, die einen kleinen Jungen ignorieren würde, nur weil er nicht unterbrochen werden wollte…„Was? Warum \emph{nicht?}“

% “Well,” said Harry Potter, “It’s going to take a while to check the whole train by hand, and we might miss the toad anyway, and if we didn’t find it by the time we’re at Hogwarts, he’d be in trouble. So what would make a lot more sense is if he went directly to the front carriage, where the prefects are, and asked a prefect for help. That was the first thing I did when I was looking for you, Hermione, although they didn’t actually know. But they might have spells or magic items that would make it a lot easier to find a toad. We’re only first-years.”
„Naja“, sagte Harry Potter, „es wird eine Weile dauern, den ganzen Zug per Hand zu durchsuchen, und wir könnten die Kröte trotzdem übersehen, und wenn wir sie nicht finden, bevor wir Hogwarts erreichen, könnte er Probleme bekommen. Was also mehr Sinn ergeben würde, wäre, wenn er direkt zum vorderen Wagen ginge, wo die Vertrauensschüler sind, und einen von ihnen um Hilfe bäte. Das war das erste, was ich getan habe, als ich nach dir gesucht habe, Hermine, auch wenn sie es dann nicht wussten. Aber sie könnten Zaubersprüche oder magische Gegenstände haben, die es wesentlich einfacher machen, eine Kröte zu finden. Wir sind ja nur Erstklässler.“

% That…\emph{did} make a lot of sense.
Das…\emph{ergab} eine Menge Sinn.

% “Do you think you can make it to the prefects’ carriage on your own?” asked Harry Potter. “I’ve sort of got reasons for not wanting to show my face too much.”
„Glaubst du, dass du alleine zum Vertrauensschülerwagen finden wirst?“, fragte Harry Potter. „Ich habe ein paar Gründe, mein Gesicht nicht allzu viel herumzuzeigen.“

% Suddenly Neville gasped and took a step back. “I remember that voice! You’re one of the Lords of Chaos! \emph{You’re the one who gave me chocolate!}”
Plötzlich japste Neville und ging einen Schritt zurück. „Ich erinnere mich an diese Stimme! Du bist einer der Herren des Chaos! \emph{Du warst der, der mir Süßigkeiten gegeben hat!}“

% What? What what \emph{what?}
Was? Was, was, \emph{was?}

% Harry Potter turned his head from the window and rose dramatically. “I \emph{never!}” he said, voice full of indignation. “Do I look like the sort of villain who would give sweets to a child?”
Harry Potter wandte sein Gesicht vom Fenster ab und stand dramatisch auf. „\emph{Niemals!}“, sagte er, seine Stimme voll der Entrüstung. „Sehe ich wie die Art von Bösewicht aus, der Kindern Süßigkeiten gibt?“

% Neville’s eyes widened. “\emph{You’re} Harry Potter? \emph{The} Harry Potter? \emph{You?}”
Nevilles Augen weiteten sich. „\emph{Du} bist Harry Potter? \emph{Der} Harry Potter? \emph{Du?}“

% “No, just \emph{a} Harry Potter, there are three of me on this train—”
„Nein, nur \emph{ein} Harry Potter, es gibt drei von mir in diesem Zug –“

% Neville gave a small shriek and ran away. There was a brief pattering of frantic footsteps and then the sound of a carriage door opening and closing.
Neville gab einen kleinen Schrei von sich und rannte weg. Es gab ein kurzes wildes Fußgetrappel und dann das Geräusch einer sich öffnenden und wieder schließenden Wagentür.

% Hermione sat down hard on her bench. Harry Potter closed the door and then sat down next to her.
Hermine setzte sich auf ihre Bank. Harry Potter schloss die Tür und setzte sich dann neben sie.

% “Can you please explain to me what’s going on?” Hermione said in a weak voice. She wondered if hanging around Harry Potter meant always being this confused.
„Kannst du mir bitte erklären, was hier vor sich geht?“, sagte Hermine mit schwacher Stimme. Sie fragte sich, ob es immer so verwirrend wäre, Zeit mit Harry Potter zu verbringen.

% “Oh, well, what happened was that Fred and George and I saw this poor small boy at the train station—the woman next to him had gone away for a bit, and he was looking really frightened, like he was sure he was about to be attacked by Death Eaters or something. Now, there’s a saying that the fear is often worse than the thing itself, so it occurred to me that this was a lad who could actually benefit from seeing his worst nightmare come true and that it wasn’t so bad as he feared—”
„Oh, naja, was passiert ist, ist, dass Fred und George und ich diesen armen kleinen Jungen auf dem Bahnsteig sitzen gesehen haben – die Frau neben ihm war für eine Zeit weggegangen, und er sah wirklich ängstlich aus, als ob er sicher wäre, dass er gleich von Todessern angegriffen würde oder so. Nun, es gibt ein Sprichwort, dass die Angst vor etwas meist schlimmer ist als die Sache selbst. Und mir schien es, dass hier ein Junge saß, der tatsächlich davon profitieren könnte, dass seine schlimmsten Alpträume wahr würden und sie nicht so schlimm wären, wie er fürchtete –“

% Hermione sat there with her mouth wide open.
Hermine saß mit weit geöffnetem Mund da.

% “—and Fred and George came up with this spell to make the scarves over our faces darken and blur, like we were undead kings and those were our grave shrouds—”
„– und Fred und George fiel dieser Zauberspruch ein, mit dem wir die Schals vor unseren Gesichtern verdunkeln und verschwimmen lassen konnten, als wäre wir untote Könige und dies wären unsere Grabtücher –“

% She didn’t like at all where this was going.
Sie mochte die Richtung, in die diese Geschichte ging, überhaupt nicht.

% “—and after we were done giving him all the sweets I’d bought, we were like, ‘Let’s give him some money! Ha ha ha! Have some Knuts, boy! Have a silver Sickle!’ and dancing around him and laughing evilly and so on. I think there were some people in the crowd who wanted to interfere at first, but bystander apathy held them off at least until they saw what we were doing, and then I think they were all too confused to do anything. Finally he said in this tiny little whisper ‘go away’ so the three of us all screamed and ran off, shrieking something about the light burning us. Hopefully he won’t be as scared of being bullied in the future. That’s called desensitisation therapy, by the way.”
„– und nachdem wir ihm all die Süßigkeiten gegeben hatten, die ich gekauft hatte, sagten wir, ’Lasst uns ihm Geld geben! Hahaha! Nimm ein paar Knuts, Junge! Nimm einen silbernen Sickel!‘, und tanzten um ihn herum, böse lachend und so weiter. Ich glaube, es gab ein paar Leute im Publikum, die eingreifen wollten, aber der Zuschauereffekt hielt sie ab, zumindest lange genug, bis sie sahen, was wir taten, und dann waren sie zu verwirrt, um irgendwas zu tun. Schlussendlich sagte er ’geht weg‘ in einem winzigen Wispern, also schrien wir drei und rannten weg, irgendwas über das ’brennende Licht‘ kreischend. Hoffentlich wird er in Zukunft nicht so eine Angst davor haben gemobbt zu werden. Man nennt das übrigens Desensibilisierung.“

% Okay, she \emph{hadn’t} guessed right about where this was going.
Okay, sie hatte \emph{nicht} richtig geraten, wohin die Geschichte ging.

% The burning fire of indignation that was one of Hermione’s primary engines sputtered into life, even though part of her \emph{did} sort of see what they’d been trying to do. “That’s awful! \emph{You’re} awful! That poor boy! What you did was \emph{mean!}”
Das brennende Feuer der Empörung, das einer von Hermines wichtigsten Antrieben war, sprang an, auch wenn ein Teil von ihr schon irgendwie \emph{sehen} konnte, was sie versucht hatten. „Das ist schrecklich! \emph{Du} bist schrecklich! Was du getan hast, war \emph{gemein!}“

% “I think the word you’re looking for is \emph{enjoyable,} and in any case you’re asking the wrong question. The question is, did it do more good than harm, or more harm than good? If you have any arguments to contribute to \emph{that} question I’m glad to hear them, but I won’t entertain any other criticisms until that one is settled. I certainly agree that what I did \emph{looks} all terrible and bullying and mean, since it involves a scared little boy and so on, but that’s hardly the key issue now is it? That’s called \emph{consequentialism,} by the way, it means that whether an act is right or wrong isn’t determined by whether it \emph{looks} bad, or mean, or anything like that, the only question is how it will turn out in the end—what are the consequences.”
„Ich glaube, dass das Wort, nach dem du suchst, ’unterhaltsam‘ ist, und in jedem Fall stellst du die falsche Frage. Die Frage ist, hat es mehr Gutes getan, als es Schaden angerichtet hat, oder andersherum? Wenn du Argumente hast, die sich um \emph{die} Frage drehen, wird es mich freuen, sie zu hören, aber ich werde keine andere Kritik akzeptieren, solange diese Frage nicht geklärt ist. Ich sehe durchaus ein, dass es schrecklich und gemein \emph{aussieht,} und als würden wir ihn mobben, da es sich um einen kleinen ängstlichen Jungen handelt, und so weiter, aber das ist kaum die Schlüsselfrage, oder? Man nennt das übrigens Konsequentialismus, es bedeutet, dass, ob eine Tat gut oder schlecht ist, nicht davon bestimmt wird, ob sie schlecht \emph{aussieht,} oder gemein oder so; die einzige Frage ist, wie es am Ende ausgeht – was die Konsequenzen sind.“

% Hermione opened her mouth to say something utterly \emph{searing} but unfortunately she seemed to have neglected the part where she thought of something to say before opening her mouth. All she could come up with was, “What if he has \emph{nightmares?}”
Hermine öffnete ihren Mund, um etwas absolut \emph{Vernichtendes} zu sagen, hatte aber unglücklicherweise den Moment verpasst, in dem sie sich überlegte, was sie sagen wollte, bevor sie den Mund öffnete. Alles was sie hervorbringen konnte, war, „Was, wenn er \emph{Alpträume} davon bekommt?“

„Ehrlich gesagt glaube ich nicht, dass er unsere Hilfe brauchte, um Alpträume zu haben, und wenn er jetzt \emph{hiervon} Alpträume hat, werden es Alpträume von schrecklichen Monstern sein, die ihm Schokolade geben und \emph{das} war im Prinzip der \emph{Sinn} der Sache.“
% “Honestly, I don’t think he needed our help to have nightmares, and if he has nightmares about \emph{this} instead, then it’ll be nightmares involving horrible monsters who give you chocolate and that was sort of the whole \emph{point}.”

Hermines Gehirn verhaspelte sich jedes Mal, wenn sie versuchte, richtig wütend zu werden. „Ist dein Leben immer so sonderbar?“, sagte sie schließlich.
% Hermione’s brain kept hiccuping in confusion every time she tried to get properly angry. “Is your life always this peculiar?” she said at last.

Harrys Gesicht strahlte vor Stolz. „Ich \emph{mache} es sonderbar. Du siehst das Produkt langer und harter Arbeit.“
% Harry Potter’s face gleamed with pride. “I \emph{make} it that peculiar. You’re looking at the product of a lot of hard work and elbow grease.”

% “So…” Hermione said, and trailed off awkwardly.
„Also…“, sagte Hermine, und verfiel in unangenehmes Schweigen.

% “So,” Harry Potter said, “how much science do you know exactly? I can do calculus and I know some Bayesian probability theory and decision theory and a lot of cognitive science, and I’ve read \emph{The Feynman Lectures} (or volume 1 anyway) and \emph{Judgment Under Uncertainty: Heuristics and Biases} and \emph{Language in Thought and Action} and \emph{Influence: Science and Practice} and \emph{Rational Choice in an Uncertain World} and \emph{Gödel, Escher, Bach} and \emph{A Step Farther Out} and—”
„Also“, sagte Harry Potter, „Wie viel Wissenschaft kennst du eigentlich? Ich kann Analysis und ich kann ein wenig Bayes'sche Wahrscheinlichkeitslehre und Entscheidungstheorie und eine Menge Cognitive Science, und ich habe Feynmans \emph{Vorlesungen über Physik (Bd. 1)} gelesen, und \emph{Judgment under Uncertainty: Heuristics and Biases} und \emph{Semantik: Sprache im Handeln und Denken} und \emph{Die Psychologie der Beeinflussung} und \emph{Rational Choice in an Uncertain World} und \emph{Gödel, Escher, Bach} und \emph{A Step Farther Out} –“

% The ensuing quiz and counter-quiz went on for several minutes before being interrupted by another timid knock at the door. “Come in,” she and Harry Potter said at almost the same time, and it slid back to reveal Neville Longbottom.
Die folgende Befragung und Gegenbefragung zog sich einige Minuten hin, bis sie von einem weiteren schüchternen Klopfen an der Tür unterbrochen wurden. „Komm rein“, sagten sie fast gleichzeitig, und die Tür glitt beiseite, um Neville Longbottom einzulassen.

% Neville \emph{was} actually crying now. “I went to the front carriage and found a p-prefect but he t-told me that prefects weren’t to be bothered over little things like m-missing toads.”
Diesmal weinte Neville wirklich. „Ich bin zum vorderen Waggon gegangen und hab einen V-Vertrauensschüler gefunden, aber er hat mir g-gesagt, dass man Vertrauensschüler nicht mit so einem Kleinkram wie fehlenden K-Kröten belästigt.“

% The Boy-Who-Lived’s face changed. His lips set in a thin line. His voice, when he spoke, was cold and grim. “What were his colours? Green and silver?”
Das Gesicht des Jungen, der lebte veränderte sich. Seine Lippen wurden zu einer feinen Linie. Als er sprach, war seine Stimme kalt und grimmig. „Was waren seine Farben? Grün und Silber?“

% “N-no, his badge was r-red and gold.”
„N-Nein, sein Abzeichen war r-rot und gold.“

% “\emph{Red and gold!}” burst out Hermione. “But those are \emph{Gryffindor’s} colours!”
„\emph{Rot und Gold!}“, brach es aus Hermine heraus. „Aber das sind \emph{Gryffindor}-Farben!“

% Harry Potter \emph{hissed} at that, a frightening sort of sound that could have come from a live snake and made both her and Neville flinch. “I \emph{suppose,}” Harry Potter spat, “that finding some first-year’s toad isn’t \emph{heroic} enough to be worthy of a \emph{Gryffindor} prefect. Come on, Neville, \emph{I’ll} come with you this time, we’ll see if the Boy-Who-Lived gets more attention. First we’ll find a prefect who ought to know a spell, and if that doesn’t work, we’ll find a prefect who isn’t afraid of getting their hands dirty, and if \emph{that} doesn’t work, I’ll start recruiting my fans and if we have to we’ll take apart the whole train screw by screw.”
Harry Potter \emph{zischte} daraufhin, ein beängstigendes Geräusch, das von einer echten Schlange hätte kommen können und sowohl sie als auch Neville zusammenzucken ließ. „Ich \emph{vermute}“, spuckte Harry Potter aus, „dass die Kröte eines Erstklässlers zu finden nicht \emph{heroisch} genug für einen \emph{Gryffindor}-Vertrauensschüler ist. Komm mit, Neville, \emph{ich} komme dieses Mal mit dir, wir wollen mal sehen, ob der Junge, der lebt, mehr Aufmerksamkeit bekommt. Zuerst finden wir einen Vertrauensschüler, der einen geeigneten Zauberspruch kennen sollte, und wenn das nicht klappt, finden wir einen Vertrauensschüler, der keine Angst hat, sich die Hände schmutzig zu machen, und wenn \emph{das} nicht klappt, rekrutieren wir meine Fans, und nehmen, wenn es sein muss, den ganzen Zug Schraube für Schraube auseinander.“

% The Boy-Who-Lived stood up and grabbed Neville’s hand in his, and Hermione realised with a sudden brain hiccough that they were nearly the same size, even though some part of her had insisted that Harry Potter was a foot taller than that, and Neville at least six inches shorter.
Der Junge, der lebt, stand auf und griff mit seiner Hand nach Nevilles, und Hermine wurde urplötzlich klar, dass sie nahezu gleich groß waren, auch wenn ein Teil von ihr darauf bestanden hatte, dass Harry Potter mindestens dreißig Zentimeter größer, und Neville mindestens fünfzehn Zentimeter kleiner wäre.

% “\emph{Stay!}” Harry Potter snapped at her—no, wait, at his \emph{trunk}—and he closed the door behind him firmly as he left.
„\emph{Bleib!}“, blaffte Harry sie an – nein, warte, er blaffte seinen \emph{Koffer} an – und er schloss die Tür fest hinter sich als er ging.

% She probably should have gone with them, but in just a brief moment Harry Potter had turned so scary that she was actually rather glad she hadn’t thought to suggest it.
Sie hätte vermutlich mitkommen sollen, aber für einen kurzen Moment war Harry Potter so angsteinflößend geworden, dass sie eigentlich ziemlich froh war, dass sie es nicht vorgeschlagen hatte.

% % The misspelling ``History: A Hogwarts'' is intentional.
% Hermione’s mind was now so jumbled that she didn’t even think she could properly read \emph{History: A Hogwarts}. She felt as if she’d just been run over by a steamroller and turned into a pancake. She wasn’t sure what she was thinking or what she was feeling or why. She just sat by the window and stared at the moving scenery.
Hermines Gedanken waren nun so durcheinander, dass sie nicht einmal mehr dachte, dass sie „\emph{Die Geschichte Hogwarts’}“ vernünftig zu Ende lesen könnte. Sie fühlte sich, als wäre sie gerade von einer Dampfwalze überfahren worden. Sie war sich nicht sicher, was sie fühlte oder warum. Sie saß einfach am Fenster und beobachtete das vorbeiziehende Landschaftsbild.

% Well, she did at least know why she was feeling a little sad inside.
Immerhin wusste sie, warum sie sich innerlich ein wenig traurig fühlte.

% Maybe Gryffindor wasn’t as wonderful as she had thought.
Möglicherweise war Gryffindor nicht so wunderbar, wie sie es gedacht hatte.

%  LocalWords:  NPC Eek Judgment
